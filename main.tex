\documentclass[a4paper,10.5pt]{bxjsarticle}
\usepackage{xltxtra}
\usepackage{zxjatype}
\setjamainfont[BoldFont=ipaexm.ttf]{ipaexm.ttf}
\setjasansfont[BoldFont=ipaexg.ttf]{ipaexg.ttf}

\usepackage{amsmath}
\usepackage{amssymb}
\usepackage{booktabs}


\title{日本の中古住宅市場におけるスマートコントラクト活用効果の検証}
\author{学籍番号:57195017 氏名:高橋 宏輝}

\begin{document}
\maketitle
\begin{abstract}
この論文では・・・
\end{abstract}

\tableofcontents

\section{序論}
この論文では、日本の中古住宅市場を対象として、スマートコントラクトの活用により取引情報が円滑に共有されることで、住宅の取引価格にどのような影響を及ぼすか検証する。

日本の住宅市場は新築住宅に偏重しており、中古住宅の流通量は少ない。2013年の国土交通白書によると日本の住宅取引全体に占める中古住宅取引の割合は14.7%であり、アメリカ、イギリス、フランスと比較して大幅に低い水準となっている。一方で、総務省の住宅・土地統計調査によると、日本の住宅総数は、2018年の段階で総世帯数に対して約879万戸超過している。居住者のいない住宅の存在は都市部でも顕著であり、首都圏の一都三県における居住者のいない住宅戸数は200万戸を超えている。しかし、中古住宅市場の流通量は少なく、これらの居住者がいない住宅ストックの活用が進んでいない。このため、日本の住宅は新築から30年程度で建物の価値はゼロとなり、土地の価値しか残らないと言われている。
\footnote{
\label{this_footnote}
住宅価格の価値の減少については***を参照。
}。

中古住宅としての販売が想定しにくい事から、日本の新築住宅市場では、将来的な他者への販売を想定せず、個人の趣味に合わせた注文住宅の割合が高い。さらに、所有期間中のメンテナンスも必要最小限が行われるのみであり、建物の価値を維持、向上する観点からのメンテナンスが行われていない。このため、中古住宅市場に品質の低い住宅が供給されることが多くなり、買い手から敬遠されるという悪循環が発生している。住宅投資を行っても30年で建物の価値が無くなるという現象は、日本の家計における資産の蓄積にもマイナスの影響を与えている。
日本全体の住宅資産の総額は、過去に行われた住宅投資の総額を大きく下回っている。一方で、アメリカでは過去の住宅投資総額を上回る住宅資産の蓄積がなされており、住宅が資産としての価値を維持している。アメリカにおいては、住宅は必要に応じて現金化することのできる資産だが、日本においては一世代で使い捨てられる消費財となっている。日本の家計が保有する資産の10%程度が住宅などの建物とされているが、資産の10%がほぼ確実に無価値になってしまうということで、家計におけるその他の消費活動にもネガティブな影響を与えていると考えられる。

\section{日本の住宅市場の特徴}
\subsection{日本の住宅市場の現状}
国土交通省が毎月公表している「不動産取引件数」によると、住宅の売買は個人-個人間での取引が最も多くなっている。

\subsection{日本の住宅行政}
日本の住宅行政の方向性を示しているのが2016年に見直しが行われた住生活基本計画である。

\subsection{リフォーム実施率}
質の良い住宅ストックの形成にはリフォームの実施が欠かせない。しかしながら、日本の住宅市場においてリフォームの実施率は高くない。この背景には原野他(2012)が指摘するように、リフォームの実施が必ずしも住宅の取引価格の上昇に結びつかない現状があると考えられる。

\subsection{住宅保証制度について}

「良質なストック住宅を形成するためには,良質であることを売主がアピールできることと,買主が質を認識できることが必要である」(藤澤2016)

\section{スマートコントラクトとは}
スマートコントラクトの概念は、分散型ブロックチェーンプラットフォームが導入されて以降、顕著に発展しているが、法的な契約ではなく、コンピュータプログラムを意味している。スマートコントラクトは、現時点では「デジタルに表現される資産をあらかじめ定められたルールに従って自動的に移転させることにすぎず、また、そのことすら実現上の課題が多い」(斉藤2017)。

スマートコントラクトの住宅取引への活用事例としては、

\section{住宅価格決定過程についての先行研究}
住宅市場は中央集権的な取引市場が存在せず、取引は局所的に行われる。
Wheaton(1990)はサーチ理論を住宅市場に拡張した。
Diaz and Jerez(2013)は、売り手があらかじめ価格を設定して、買い手がそれを見て購入を申し込むという仕組みを導入し、住宅市場の流動性が高いときには住宅価格も高くなるモデルを提唱した。

日本の既存住宅市場における情報開示量が成約価格や成約率、売り出し価格に与える影響について、藤澤(2016)は住宅の質情報がわかりやすく全開示情報となれば良質な維持管理へのインセンティブを与えると指摘している。
\section{理論モデルの構築}
住宅の品質を新築時に決まるものと、所有期間における維持管理によって決まるものに分類し、売主は住宅を売却する段階で品質を変更することはできず、品質を所与として価格設定を行うものとする。売主は住宅を所有するもの中から$\lambda_{d}$の確率で発生するとする。

藤澤(2016)では、売主と買主はそれぞれ同数として、売主、買主の探索の摩擦を考慮していない。しかし、現実の住宅取引においては、売主と買主の数は等しいとは限らない。Duffie(2007)は売主の数が買主よりも少ない場合、取引価格は減少すると指摘している。そこで本研究では、住宅の所有者のうち、一定の確率で住宅を所有する意思が低下し、

\section{住宅市場へのスマートコントラクトの導入}
\subsection{解くべき問題}
斉藤(2016)は不動産取引が売主、買主などの取引関係者が一同に会して行われている背景として、「意思確認の効率化と、契約不履行による様々なリスクを避けるためといった理由があると考えられる」と指摘した上で、意思確認とリスク回避の手段としてスマートコントラクトを用いて売買を成立させることができるモデルを紹介している。

\subsection{住宅市場の関与者}
売主、買主に加えて保険会社、銀行が関与すると想定される。

\subsection{コントラクト}
\begin{itemize}
\item デジタル通貨
\item 住宅アセット

土地及び土地に付属する建物の所有権をデジタル資産として表現したものである。他の住宅アセットと識別するために住所情報を持っている。
\item 住宅売買契約
\item 保険契約

\end{itemize}

\subsection{マイナーのインセンティブ or Gasについて}

\section{マルチエージェントシミュレーションによる検証}
\section{モデルの評価}
\section{関連研究}
\section{結論}

\begin{thebibliography}
\bibliography{export}
\bibliographystyle{jplain}
\end{thebibliography}

\end{document}
