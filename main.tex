\documentclass[a4paper,10.5pt]{jlreq}
%\usepackage{xltxtra}
%\usepackage{zxjatype}
%\setjamainfont[BoldFont=ipaexm.ttf]{ipaexm.ttf}
%\setjasansfont[BoldFont=ipaexg.ttf]{ipaexg.ttf}

\usepackage{amsmath}
\usepackage{amssymb}
\usepackage{booktabs}
\usepackage{cite}


\title{日本の中古住宅市場におけるスマートコントラクト活用効果の検証}
\author{
学籍番号:57195017 氏名:高橋 宏輝
\\ゼミ名称:ブロックチェーンと分散ファイナンス演習
\\主査:斉藤 賢爾教授 副査:中里 大輔教授}

\begin{document}
\date{}
\maketitle
\begin{abstract}
この論文では・・・
\end{abstract}

\tableofcontents

\section{序論}
この論文では、日本の中古住宅市場を対象として、スマートコントラクトの活用により取引情報が円滑に共有されることで、住宅の取引価格にどのような影響を及ぼすか検証する。

日本の住宅市場は新築住宅に偏重しており、中古住宅の流通量は少ない。2013年の国土交通白書によると日本の住宅取引全体に占める中古住宅取引の割合は14.7%であり、アメリカ、イギリス、フランスと比較して大幅に低い水準となっている。一方で、総務省の住宅・土地統計調査によると、日本の住宅総数は、2018年の段階で総世帯数に対して約879万戸超過している。居住者のいない住宅の存在は都市部でも顕著であり、首都圏の一都三県における居住者のいない住宅戸数は200万戸を超えている。しかし、中古住宅市場の流通量は少なく、これらの居住者がいない住宅ストックの活用が進んでいない。このため、日本の住宅は新築から30年程度で建物の価値はゼロとなり、土地の価値しか残らないと言われている\footnote{住宅価格の価値の減少については***を参照。}。

中古住宅としての販売が想定しにくい事から、日本の新築住宅市場では、将来的な他者への販売を想定せず、個人の趣味に合わせた注文住宅の割合が高い。さらに、所有期間中のメンテナンスも必要最小限が行われるのみであり、建物の価値を維持、向上する観点からのメンテナンスが行われていない。このため、中古住宅市場に品質の低い住宅が供給されることが多くなり、買い手から敬遠されるという悪循環が発生している。住宅投資を行っても30年で建物の価値が無くなるという現象は、日本の家計における資産の蓄積にもマイナスの影響を与えている。
日本全体の住宅資産の総額は、過去に行われた住宅投資の総額を大きく下回っている。一方で、アメリカでは過去の住宅投資総額を上回る住宅資産の蓄積がなされており、住宅が資産としての価値を維持している。アメリカにおいては、住宅は必要に応じて現金化することのできる資産だが、日本においては一世代で使い捨てられる消費財となっている。日本の家計が保有する資産の10%程度が住宅などの建物とされているが、資産の10%がほぼ確実に無価値になってしまうということで、家計におけるその他の消費活動にもネガティブな影響を与えていると考えられる。

\section{日本の住宅市場の特徴}
\subsection{日本の住宅市場の現状}
国土交通省が毎月公表している「不動産取引件数」によると、住宅の売買は個人-個人間での取引が最も多くなっている。

\subsection{日本の住宅行政}
日本の住宅行政の方向性を示しているのが2016年に見直しが行われた住生活基本計画である。

\subsection{リフォーム実施率}
質の良い住宅ストックの形成にはリフォームの実施が欠かせない。しかしながら、日本の住宅市場においてリフォームの実施率は高くない。この背景には原野他(2012)が指摘するように、リフォームの実施が必ずしも住宅の取引価格の上昇に結びつかない現状があると考えられる。

\subsection{住宅保証制度について}

「良質なストック住宅を形成するためには,良質であることを売主がアピールできることと,買主が質を認識できることが必要である」(藤澤2016)

\section{スマートコントラクトとは}
スマートコントラクトは、ブロックチェーン上で資産の移転などの処理を行うプログラムであり、法的な契約ではない。従来の契約行為においては、契約当事者が債務を履行したかの確認は自動化されておらず、送金等の手続きは契約行為と別に行う必要があった。スマートコントラクトにおいては、契約当事者が債務を履行できる状態にあるかをプログラムが判定し、債務の履行はブロックチェーン上の価値の移転によって自動的に行われるようプログラムすることができる。
スマートコントラクトの概念は、Ethereumなどの分散型ブロックチェーンプラットフォームが導入されて以降、顕著に発展しているが、現時点では「デジタルに表現される資産をあらかじめ定められたルールに従って自動的に移転させることにすぎず、また、そのことすら実現上の課題が多い」(斉藤2017)
%\cite{RefWorks:doc:5fa4bedfe4b0e028cce8d973}
との指摘がされている。
スマートコントラクトで実行された内容はブロックチェーン上に不可逆的に記載されるため、取り消すことができない。「これはCode is lawとよく言われる概念であり、スマートコントラクトにおいては、ブロックチェーン上に書かれたプログラムが、契約書であり、裁判所でもあることになるなる。」


スマートコントラクトが実行されるブロックチェーンネットワークには、パブリックとプライベートの区別がある。パブリックブロックチェーンは誰でも読み込み、書き込みすることができる。プライベートブロックチェーンではユーザー毎に異なる権限設定を行える。Ethereum、Bitcoinはパブリックブロックチェーンであり、


スマートコントラクトの住宅取引への活用事例としては、米国のPROPY\footnote{https://propy.com}がある。同社は不動産の売買取引において、第三者として売主から権利証書、買主から売買代金を預かり、行政の登記手続きが終了するまでの間ブロックチェーン上に取引の情報を記録し、登記手続きが完了すると権利証書と売買代金の移転を行うというサービスをオンライン上で提供している。同じく米国のRentberry\footnote{https://rentberry.com}は、米国およびヨーロッパの都市部において賃貸住宅の仲介サービスを提供しており、入居者はオンライン上で全ての契約手続きを完了させることができる。さらに、Rentberryでは、敷金の支払いが難しい入居者が第三者から敷金の借り入れを受けることができるP2Pレンディングサービスも提供している。


\section{住宅価格決定過程についての先行研究}
住宅市場は中央集権的な取引市場が存在せず、取引は局所的に行われる。
Wheaton(1990)はサーチ理論を住宅市場に拡張した。
Diaz and Jerez(2013)は、売り手があらかじめ価格を設定して、買い手がそれを見て購入を申し込むという仕組みを導入し、住宅市場の流動性が高いときには住宅価格も高くなるモデルを提唱した。

日本の既存住宅市場における情報開示量が成約価格や成約率、売り出し価格に与える影響について、藤澤(2016)は住宅の質情報がわかりやすく全開示情報となれば良質な維持管理へのインセンティブを与えると指摘している。
\section{理論モデルの構築}
住宅の品質を新築時に決まるものと、所有期間における維持管理によって決まるものに分類し、売主は住宅を売却する段階で品質を変更することはできず、品質を所与として価格設定を行うものとする。売主は住宅を所有するもの中から$\lambda_{d}$の確率で発生するとする。

藤澤(2016)では、売主と買主はそれぞれ同数として、売主、買主の探索の摩擦を考慮していない。しかし、現実の住宅取引においては、売主と買主の数は等しいとは限らない。Duffie(2007)は売主の数が買主よりも少ない場合、取引価格は減少すると指摘している。そこで本研究では、住宅の所有者のうち、一定の確率で住宅を所有する意思が低下し、

\section{住宅市場へのスマートコントラクトの導入}
\subsection{解くべき問題}
%斉藤(2016)は不動産取引が売主、買主などの取引関係者が一同に会して行われている背景として、「意思確認の効率化と、契約不履行による様々なリスクを避けるためといった理由があると考えられる」と指摘した上で、意思確認とリスク回避の手段としてスマートコントラクトを用いて売買を成立させることができるモデルを紹介している。
住宅取引における問題は、売主と買主の間で住宅の品質に関する情報が共有されておらず、適切な値付けが行われないことである。買主としては住宅の品質が分からない以上、品質が低い場合にも損をしないような値段づけを行うことになる。これに対して、売主、買主以外の第三者が、取引対象の住宅に対して、その品質を調査し証明を行い、その情報をブロックチェーン上に記録することを検討する。これにより、買主は住宅の品質を確認することができるようになり、ブロックチェーン上にある他の売買事例を参考にして、購入を検討している住宅に対するより適正な値付けを行うことができる。

\subsection{住宅取引の関与者}
売主、買主に加えてインスペクター、保険会社%、銀行
が関与すると想定される。これらがそれぞれに外部所有アカウントを持っている。今回検討するモデルでは住宅自体がアカウントをもち、品質情報を持っていることを考える。
\begin{itemize}
	\item 売主
    \item 買主
    \item インスペクター
    \item 保険会社
    %\item 銀行
	\item 住宅アセット	
	\par 土地及び土地に付属する建物の所有権をデジタル資産として表現したものである。他の住宅アセットと識別するために住所情報を持っている。また過去の取引価格と住宅瑕疵保険の等級情報を持っており、住所も含めてこれらの情報は第三者から参照可能となっている。
	\item 住宅売買契約
	\item 保険契約

\end{itemize}


\subsection{ブロックチェーンネットワークの選択}


\subsection{コントラクト}

\subsection{マイナーのインセンティブ or Gasについて}

\section{マルチエージェントシミュレーションによる検証}
\section{モデルの評価}
\section{関連研究}
\section{結論}
品質情報を含めたスマートコントラクトにより住宅取引を行い、価格情報等の取引情報をブロックチェーン上に公開して情報公開を進めることは、中古住宅取引の活性化に繋がり、中古住宅の取引価格を引き上げると考えられる。

\bibliographystyle{junsrt}
\bibliography{export}

\end{document}
