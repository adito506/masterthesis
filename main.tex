\documentclass[a4paper,10pt]{bxjsarticle}
\usepackage{xltxtra}
\usepackage{zxjatype}
\setjamainfont[BoldFont=ipaexm.ttf]{ipaexm.ttf}
\setjasansfont[BoldFont=ipaexg.ttf]{ipaexg.ttf}

\usepackage{amsmath}
\usepackage{amssymb}
\usepackage{booktabs}


\title{日本の中古住宅市場におけるスマートコントラクト活用効果の検証}
\author{学籍番号:57195017 氏名:高橋 宏輝}

\begin{document}
\maketitle

\section{序論}
この論文では、日本の中古住宅市場を対象として、スマートコントラクトの活用により取引情報が円滑に共有されることで、住宅の取引価格にどのような影響を及ぼすか検証する。

\section{日本の住宅市場の特徴}
日本の住宅市場は新築住宅に偏重しており、中古住宅の流通量は少ない。2013年の国土交通白書によると日本の住宅取引全体に占める中古住宅取引の割合は14.7%であり、アメリカ、イギリス、フランスと比較して大幅に低い水準となっている。一方で、総務省の住宅・土地統計調査によると、日本の住宅総数は、2018年の段階で総世帯数に対して約879万戸超過しており、これらの住宅には誰も居住していない。居住者のいない住宅の存在は都市部でも顕著であり、首都圏の一都三県における居住者のいない住宅戸数は200万戸を超えている。中古住宅市場が未発達であることもあり、日本の住宅は新築から30年程度で建物の価値はゼロとなり、土地の価値しか残らないと言われている。また、中古住宅としての販売が想定しにくい事から、将来的な販売を前提としない個人の趣味に合わせた注文住宅の割合も多い。さらに、所有期間中のメンテナンスも必要最小限が行われるのみであり、建物の価値を維持、向上する観点からのメンテナンスはあまり行われていない。このため、中古住宅市場に供給される物件も質の低い物件が多くなり、ますます買い手から敬遠されるという悪循環が発生している。住宅投資を行っても30年で建物の価値が無くなるという現象は、日本の家計における資産の蓄積にもマイナスの影響を与えている。日本全体の住宅資産の総額は、過去に行われた住宅投資の総額を大きく下回っている。一方で、アメリカでは過去の住宅投資総額を上回る住宅資産の蓄積がなされており、住宅が資産としての価値を維持している。アメリカにおいては、住宅は必要に応じて現金化することのできる資産だが、日本においては一世代で使い捨てられる消費財となっている。日本の家計が保有する資産の10%程度が住宅などの建物とされているが、資産の10%がほぼ確実に無価値になってしまうということで、家計におけるその他の消費活動にもネガティブな影響を与えていると考えられる。

\section{スマートコントラクトとは}
\section{住宅価格決定過程についての先行研究}
\section{価格を最大化するパラメーター}
\section{モデルを実現するスマートコントラクトのアルゴリズム}
\section{マルチエージェントシミュレーションによる検証}
\section{モデルの評価}
\section{関連研究}
\section{結論}

% \begin{table}[h]
% \centering
% \begin{tabular}{cc}
% \toprule
% 名前 & 色 \\
% \midrule
% りんご & 赤または緑 \\
% みかん & 橙\\
% \bottomrule
% \end{tabular}
% \end{table}

% \begin{align}
% f(x) = ax + b
% \end{align}

%\begin{thebibliography}
% \bibitem{}
%\end{thebibliography}
\end{document}
